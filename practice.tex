\documentclass{article}
\usepackage[utf8]{inputenc}

\usepackage{kotex}
\usepackage{amsmath}
\usepackage{verbatim}
\usepackage{amsfonts}
\usepackage{amssymb}
\usepackage{url}
\usepackage{color}
\usepackage{slashbox}
\usepackage{multirow}
\usepackage{tabularx}


\begin{document}

\title{LaTeX Practice}
\author{박석훈 / 학생 / 사회학과 }
\date{March 2022}

\maketitle

\section{Introduction}

\begin{eqnarray}
f(x) & = & \cos x \\
f'(x) & = & -\sin x
\end{eqnarray}

{\setlength\arraycolsep{2pt}
\begin{eqnarray}
f(x) & = & \cos x \\
f'(x) & = & -\sin x
\end{eqnarray}}


$ A \, A \quad A \qquad A $

Cahn--Hilliard

$\boldsymbol{\mu}$

\begin{eqnarray}
f(x) = \cos x \label{fc}
\end{eqnarray}

\eqref{fc}

\begin{align}
    \begin{split}
        \phi_t&=\Delta \phi, \\
        \phi_t&=\frac{\phi-\phi^3}{\epsilon^2}.\label{gv_eq}
    \end{split}
\end{align}

%$\Big( (x+1) (x-1) \Big) ^{2}$\\
$\big(\Big(\bigg(\Bigg($\quad $\big\}\Big\}\bigg\}\Bigg\}$
 \quad $\big\|\Big\|\bigg\|\Bigg\| \quad
$\big/\Big/\bigg/\Bigg/$ \quad $\big| \Big| \bigg| \Bigg|

\begin{displaymath}
\mathop{\mathrm{corr}}(X, Y) = \frac{\displaystyle
\sum_{i=1}^n(x_i-\overline x) (y_i-\overline y)}
{\displaystyle\left[ \sum_{i=1}^n(x_i-\overline x)^2
\sum_{i=1}^n(y_i-\overline y)^2 \right]^{1/2}}
\end{displaymath}

\begin{displaymath}
\mathop{\mathrm{corr}}(X, Y)= \frac{\displaystyle
\sum_{i=1}^n(x_i-\overline x) (y_i-\overline y)}
{\displaystyle\biggl[ \sum_{i=1}^n(x_i-\overline x)^2
\sum_{i=1}^n(y_i-\overline y)^2 \biggr]^{1/2}}
\end{displaymath}

$ log(x), \log(x), sin x, \sin x $

$ \langle x, y \rangle$

{\tiny tiny}, \\
{\scriptsize scriptsize},
{\footnotesize footnotesize}, {\small small}
{\normalsize normalsize}, {\large large}, {\Large Large},
{\LARGE LARGE}, {\huge huge}, {\Huge Huge}

\begin{equation}
f(x) = \begin{cases}
            0 & \text{if ~$x=0$,} \\
            x & \text{otherwise.} 
      \end{cases}
\end{equation}

\begin{equation}
1+(-1)^n=\begin{cases}
			0, & \text{if $n$ odd}\\
            2, & \text{otherwise}
		 \end{cases}
\end{equation}


\begin{equation}
    \frac{\max_{1 \leq m \leq M} | \phi^n_m|}
    {\max_{1 \leq m \leq M}|\phi^0_m|} =
    \frac{\displaystyle\max_{1\leq m \leq M} |\phi^n_m|}
    {\displaystyle\max_{1 \leq m \leq M}|\phi^0_m|}
\end{equation}

\verb+eg_pchip.m+ will generate the following figure.
\verb#eg_pchip.m# will generate the following figure.

\begin{verbatim}
    10 PRINT "HELLO WORLD";
    20 GOTO 10
\end{verbatim}

ADD $a$ and $b$ to get $c$.
Or, using a more mathematical approach:
\begin{displaymath}
c=a+b
\end{displaymath}
or you can type less with:
\[ a+b = c \]

\begin{equation}
    \epsilon > 0 \label{eq:eps}
\end{equation}
From Eq. (\ref{eq:eps}), we can see it is positive.

$ \lim_{x \to 1}(x+1) = 2$



\begin{displaymath}
\lim_{x \to 1}(x+1) = 2    
\end{displaymath}

$ a_1, x^2, e^{-\alpha}$

$\sqrt{x+y}, ~\sqrt[3]{2}$

\begin{displaymath}
\frac{x^2}{k+1}
\end{displaymath}

\begin{displaymath}
1 + \left( \frac{1}{1 - x^2} \right) ^3
\end{displaymath}

$a^x+y \neq a^{x+y}, e^{x^2} \neq {e^x}^2, a^3_{ij}$,
$x^{\frac{2}{k+1}}$

\begin{displaymath}
x^{2} \geq 0\qquad \textrm{for all }x\in\mathbf{R}
\end{displaymath}

\begin{displaymath}
x^{2} \geq 0\qquad \textrm{for all }x\in\mathbb{R}
\end{displaymath}

$\scriptstyle (\verb+\mbox+ \mbox{의 경우})~C_i$
$\mbox{ such that } \forall i$
$(\verb+\text+ \mbox{의 경우})~C_i \text{such that } \forall i$
$\alpha, \beta, \gamma, \delta, \epsilon, \varepsilon, \zeta, \eta$

$\theta, \vartheta, \kappa, \lambda, \mu, \nu, \xi, \pi, \rho$

$\sigma, \tau, \phi, \varphi, \chi, \psi, \omega$
$\le, \ll, \subset, \subseteq, \in, \ge, \gg, \supset, \subseteq$
$\ni, \|, \equiv, \doteq, \sim, \simeq, \approx, \cong$
$\propto, \perp, \ne, \pm, \cdot, \times, \cup, \oplus, \otimes$
$\bigtriangleup, \lhd, \mp, \div, \setminus, \cap, \bigcirc$
$\bigtriangledown, \rhd, \triangleleft, \triangleright, \star$
$\Box, \ast, \circ, \bullet, \diamond, \dagger, \ddagger, \dots, \cdots$
$\sum, \prod, \int, \oint, \bigcup, \bigcap, \vdots, \ddots$
$\nabla, \partial, \infty, \prime, \exists, \forall$
$\because, \therefore$
$\$, \&, \%, \#, \_, \{, \}, \backslash$
It is $35\, ^{\circ}\mathrm{C}$.
$y = x^2, y'=2x, y''=2$

\begin{displaymath}
\vec a\quad\overrightarrow{AB}
\end{displaymath}

$ \underbrace{ a+b+\cdots+z }_{26} $

\begin{displaymath}
\lim_x{x \rightarrow 0}, \sum_{i=1}^{n},
\int_{0}^{\frac{\pi}{2}}, \prod_\epsilon
\end{displaymath}

\begin{displaymath}
\sum_{\substack{0<i<n \\ 1<j<n}}
P(i, j) = \sum_{\begin{subarray}{1} i\in I \\ i<j<m
\end{subarray}} Q(i, j)
\end{displaymath}

\begin{displaymath}
{a, b, c}\neq\{a, b, c\}
\end{displaymath}

% a \bmod b, x \equiv a \pmod{b} 

\begin{displaymath}
\binom{n}{k} \qquad \mathrm{C}_n^k \qquad _n\mathrm{C}_r
\end{displaymath}

\begin{displaymath}
\int f_N(x) \stackrel{!}{=} 1
\end{displaymath}

\newcommand{\ud}{\mathrm{d}
\begin{displaymath}
\int\!\!\!\int_{D} g(x, y) \, \ud x\, \ud y
\end{displaymath}
instead of
\begin{displaymath}
\int\int_{D} g(x, y)\ud x \ud y
\end{displaymath}

\newcommand{\ud}{\mathrm{d}}
\begin{displaymath}
\iint_{D} \, \ud x \, \ud y
\end{displaymath}

${\displaystyle \int f^{-1}(x-x_a)\, dx},
{\textstyle \int f^{-1}(x-x_a)\, dt},
{\scriptstyle \int f^{-1}(x-x_a)\,dx},
{\scriptscriptstyle \int f^{-1}(x-x_a)\,dx}$

\begin{displaymath}
{}^{12}_{\phantom{1}6}\textrm{C}, {}^{12}_{6}\textrm{C}
\end{displaymath}

\begin{displaymath}
\Gamma_{ij}^{\phantom{ij}k}, \Gamma_{ij}^{k}
\end{displaymath}

\begin{displaymath}
\mathbf{X} = \left( \begin{array}ccc
    x_{11} & x_{12} & \ldots \\
    x_{21} & x_{22} & \ldots \\
    \vdots & \vdots & \ddots
\end{array} \right)
\end{displaymath}

\begin{eqnarray}
\sin x & = & x -\frac{x^3}{3!} + \frac{x^5}{5!} \nonumber \\
& & -\frac{x^7}{7!}+\cdots
\end{eqnarray}

http://math.korea.ac.kr/$\sim$cfdkim/

\label{marker}, \ref{marker}, \pageref{marker}

A reference to this subsection \label{sec:this} looks like:
``see section \ref{sec:this} on page \pageref{sec:this}.''

\footnote{footnote text}

Footnotes\footnote{This is a footnote.} are often used by people using \LaTeX.

\emph{If you use emphasizing inside a piece of emphasized text, then \LaTeX{} uses the \emph{normal} font for emphasizing.}

\textit{You can also \emph{emphasize} text if it is set in italics,}

{\color{blue} The text is blue.}


\flushleft
\begin{enumerate}
    \item You can mix the list environments to your taste:
\begin{itemize}
    \item But it might start to look silly.
    \item[-] With a dash.}    
\end{itemize}
\item Therefore remember:
\begin{description}
\item[Stupid] things will not become smart because they are in a list.
\item[Smart] things, though, can be presented beautifully in a list.
\caption{caption1}
\end{description}
\end{enumerate}



\tableofcontents


\section{...}
\subsection{...}
\subsubsection{...}

\maketitle

\title{...}
\author{...}
\date{...}


\begin{table}[h]
\caption{Comparison of $1_2$ error and (CPU time).}
\begin{center}
    \begin{tabular}{|c|c|c|}
    \hline
    Mesh & Bi-CGStab & OSM \\
    \hline
    $32 \times 32$  & 0.0161~~(0.33)~ & 0.0160~(0.31) \\
    $64 \times 64$  & 0.0126~~(1.67)~ & 0.0126~(1.17) \\
    $128 \times 128$  & 0.0078~~(6.42)~ & 0.0076~(4.67) \\
    $256 \times 256$  & 0.0089~~(90.50)~ & 0.0087~(19.34) \\
    \hline
    \end{tabular}
\end{center} \label{tab1}
\end{table}

\begin{table}[h]
\caption{Comparison of $1_2$ error and (CPU time).}
\begin{center}
    \begin{tabular}{|c|c|c|}
    \hline
    \backslashbox{Mesh}{Method} & Bi-CGStab & OSM \\
    \hline
    $32 \times 32$  & 0.0161~~(0.33)~ & 0.0160~(0.31) \\
    $64 \times 64$  & 0.0126~~(1.67)~ & 0.0126~(1.17) \\
    $128 \times 128$  & 0.0078~~(6.42)~ & 0.0076~(4.67) \\
    $256 \times 256$  & 0.0089~~(90.50)~ & 0.0087~(19.34) \\
    \hline
    \end{tabular}
\end{center}
\end{table}

\begin{table}[h]
\caption{Option price}
\begin{center}
    \begin{tabular}{|c|*{4}{c|}c|}
    \hline Underlying assets & \multicolumn{4}{|c|}{Option Price(OSM)} & Option price(Exact) \\
    \cline{2-5}
    $(x, \ y)$ & $h=2$ & $h=1$ & $h=0.5$ & $h=0.25$ \\
    \hline
    (90, 90) & 1.5391 & 1.5325 & 1.5287 & 1.5267 & 1.5475 \\
    \hline
    (100, 100) & 3.1319 & 3.1403 & 3.1442 & 3.1461 & 3.1935 \\
    \hline
    (110, 110) & 4.9759 & 4.9963 & 5.0064 & 5.0115 & 5.0889 \\
    \hline
    \end{tabular}
\end{center}
\end{table}

\begin{table}[h]
\begin{center}
    \caption{Relative errors} \label{tab_h}
    \begin{tabular}{|c|*{6}{c|}c|}
    \hline& \multicolumn{3}{c|}{$\Gamma_{xx}$} & \multicolumn{3}{c|}{$\Gamma_{xy}$} \\
    \cline{2-7}
    $h$ & 2nd order & 4th order & 6th order & 2nd order & 4th order & 6th order \\
    \hline 1.000000 & 2.067e-3 & 1.355e-3 & 1.354e-3 & 1.628e-3 & 7.011e-4 & 6.985e-4 \\
    \hline 0.500000 & 5.172e-4 & 3.387e-4 & 3.387e-4 & 4.075e-4 & 1.748e-4 & 1.747e-4 \\
    \hline 0.250000 & 1.293e-4 & 8.467e-5 & 8.466e-5 & 1.019e-4 & 4.368e-5 & 4.367e-5 \\
    \hline 0.125000 & 3.233e-5 & 2.117e-5 & 2.117e-5 & 2.548e-5 & 1.092e-5 & 1.092e-5 \\
    \hline 0.062500 & 8.014e-6 & 5.212e-6 & 5.209e-6 & 6.345e-6 & 2.698e-6 & 2.694e-6 \\
    \hline 0.031250 & 2.021e-6 & 1.283e-6 & 1.258e-6 & 1.676e-6 & 7.938e-7 & 8.082e-7 \\
    \hline 0.015625 & 1.312e-7 & 1.066e-7 & 1.232e-7 & 2.696e-7 & 9.799e-10 & 1.646e-8 \\
    \hline    
    \end{tabular}
    \label{tab:my_label}
\end{center}    
\end{table}



\begin{table}[htbp]
\caption{Relative errors}
\begin{center}
\begin{tabular}{cc|ccccccc}
\toprule
    &\backslashbox{error}{$h$} & $1$ & 0.5 & 0.25 & 0.125 & 0.0625 & 0.03125 \\
\midrule
\multirow{3}{\newlength\LL\settowidth\LL{10}}{$\Gamma_{xx}$}
         & 2nd order & 2.067e-3 & 5.172e-4 & 1.293e-4 & 3.233e-5 & 8.083e-6 & 2.021e-6\\
         & 4th order & 1.355e-3 & 3.387e-4 & 8.468e-5 & 2.117e-5 & 5.292e-6 & 1.323e-6\\
         & 6th order & 1.354e-3 & 3.387e-4 & 8.467e-5 & 2.117e-5 & 5.292e-6 & 1.323e-6\\
    \midrule
    \multirow{3}[\newlength\LL\settowidth\LL{10}]{$\Gamma_{xy}$}
         & 2nd order & 1.628e-3 & 4.075e-4 & 1.019e-4 & 2.548e-5 & 6.369e-6 & 1.592e-6\\
         & 4th order & 7.011e-4 & 1.748e-4 & 4.368e-5 & 1.092e-5 & 2.730e-6 & 6.823e-7\\
         & 6th order & 6.985e-4 & 1.747e-4 & 4.367e-5 & 1.092e-5 & 2.730e-6 & 6.823e-7\\
    \bottomrule         
    \end{tabular}
    \end{center}
\end{table}

\multirow{3}{\newlength\LL\settowidth\LL{10}}{$\Gamma_{xy}$}




\end{document}


